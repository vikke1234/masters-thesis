\chapter{Methodology}
\section{Introduction}
In this chapter we'll give an overview of how the research was conducted on
computational photography. This thesis aims to give an overview of the field,
giving a background into various hardware that a camera uses. How these
cameras can be programmed, and finally we'll go over recent trends in the
field.

The methodology is to answer the following questions: How do ISPs work to
enchance images? How do sensors work? What role do camera APIs have? What
are issues currently faced when trying to use cameras? What are current trends
in the field?

\section{Data collection}
\subsection{Methods}
Based on earlier expertise, we collected data in many different ways, firstly
we looked at standards for various technologies. For comparing the APIs we
looked at the docs of the APIs, we also provide some sample programs that the
reader can run for simple uses that can be run along with build instructions.
These programs are quite platform specific.

We also conducted several interviews, which largely are anonymized. We will
mention what position the person was in and which field. These interviews were
designed to get an understanding of what the pain points currently are when
programming cameras. We looked for potential interviewees on various forums
related to robotics, development, as well as asked various contacts in related
fields if they knew interested parties. The full text of interviews can be
found in \cref{appendix:interviews}, unless specifically mentioned that it is
not available due to request.

TODO computational photography papers

\subsection{Limitations}
We try to give a comprehensive view of everything, but due to the nature of the
field not everything is public information. The thesis can not always give a
detailed description of everything, especially when dealing with hardware such
as ISPs. Additionally we will only give a high level look into what these
do, details of the inner workings of these devices can be found in the
specification if it's available.

