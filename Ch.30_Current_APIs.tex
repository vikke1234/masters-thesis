\chapter{Current API's\label{currentAPIs}}
\section{Argus}
%https://docs.nvidia.com/jetson/l4t-multimedia/group__LibargusAPI.html

Argus is a camera API that's developed by Nvidia. Argus runs only on Nvidia
hardware, such as the Jetson Orin Nano Developer Kit.

Argus is roughly based on V4L2 though effectively a re-implementation. At the
time Argus was created, V4L2 was still in an early stage. It di not supporting
a variety of features such as multicamera setups etc. hence Argus was created
in order to add the missing features easily. These days V4L2 and Argus are very
close though in later chapters we'll give an overview of what pros and cons
each has.

The reason Argus doesn't use V4L2 is largely a historical one at this point.
At the time Argus was created V4L2 was in an early stage, it had very limited
features. Nvidia created Argus in order to leviate these issues but ended up
with an API that is very similar to V4L2. Automotive for example needed support
for multi camera use cases, this was something that wasn't supported in V4L2 at
the time.

Over time, the API has aged a little though. Today it's still very dependent on
EGL. EGL is a very dated API, like OpenGL it's largely in maintenance mode
today.

\section{libcamera}
libcamera is an open-source (with binary blobs for some cameras) C++ embedded
camera framework that supports a large number of complex cameras such as the
IMX 219, 477 and many more. It supports multiple encoders to receive images in
for example PNGs/raw images. The primary target for libcamera is Arm processors
in the form of Raspberry PI's, Chrome OS and Android though many other
architectures are also supported.

Before libcamera, the way to use camera sensors was very complex, often
requiring it's own MCU. libcamera moved this almost completely to userspace,
trivializing many applications. libcamera supports a variety of IPA's
for 3A (Auto focus, Auto exposure, Auto white balancing). Camera
configuration is done using V4L2, libcamera being essentially an extension for
it.

\section{HAL3}
Hardware Abstraction Layer 3 (HAL3) is probably the most well known one that
will be discussed in this thesis. It's included in each Android phone, it has
remarkable features for a mobile phone camera. It was created to bridge the gap
between the higher level the API camera2 and the lower level hardware API's. It
allows for more modification than camera2, while requiring more work to manage.

\section{Kamaros}
Kamaros is a new API
