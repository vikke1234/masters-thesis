\chapter{History and Background\label{history}}
\section{Background}
This section will introduce the necessary concepts needed for understanding the
thesis.

\subsection{Sensor technology}
In the end, sensor technology is just some material that can convert photons
into electrical charge. In this section we'll cover two common ways this is
done, as well as a brief list of pros and cons for each.

\subsubsection{CMOS Sensors}
\textit{Complementary Metal Oxide Semiconductor} (CMOS) sensors are very old
sensors, being around since ~1960. Over the years they've become competetive
with \textit{Charged Couple Devices} (CCDs) which are a competetive technology.
While CCDs have a better image quality, it requires much more power than CMOS
which can be up to 100 times less power hungry \cite{CMOSReview}. This means
that in many mobile devices and many other low power devices wants CMOS
over CCDs.

TODO: add brief description of CMOS

This thesis will not provide a very in depth view into how exactly the
technology works. Curious readers can have a look at \cite{CMOSReview} as well
as \cite{ieeeCMOS}.

\subsubsection{CCD Sensors}
TODO

\subsection{Image Signal Processors}
Image Signal Processors (ISPs) are a highly secretive piece of hardware. There
are few systems that say they even have one, and even fewer that explain how
they work. To understand how ISPs work, we'll be using the Raspberry Pi
\cite{raspberrypiTuningGuide}, it's the only one that is open source (except
for the RTL code). In this section we'll give an overview of what ISPs
typically do and how they function.

So ISPs, what exactly are they? As the name suggests, they process image signals.
When an image come from the sensor the signal (image) contains a lot of
redundant information. The sensor is also also not quite calibrated to the real
world environment, a lot of things are simply wrong with it. Correcting these
is an expensive process, so much so that there's a HW block in camera systems
that does this for you.

The isp has a couple steps, most work something along the lines of

\begin{enumerate}
    \item Receive raw sensor data, often a Bayer image or similar
    \item Demosaic the image, i.e. extracting the pixel data from the raw image.
          The is often done using a Bayer like filter, many exist though they
          all work in the same way.
    \item Begin the autocontrol algorithms


\end{enumerate}


\section{History}
In the early days cameras were quite simple, there effectively was just the
"press button" followed by receiving a picture. This process did not allow for
much customization. This was the case with most cameras such as webcams where
the camera sensors were "smart sensors". This meant that the . Even the
integrated laptop ones were just USB devices. Linux had initially had lacking
support for this, but in the early 2000s Video For Linux (V4L, later V4L2 for
version 2) was developed. This covered most of the use cases at the time.

% TODO: cite frankencamera
In 2009 the Nokia N900 was released, this was a Linux based phone that was not
like most cameras at the time. It provided interfaces for customizing just
about everything. From ISPs to Image Processing Algorithms (IPA's), this meant
that the current way camera API's worked was no longer sufficient. Enter
Frankencamera\cite{adams2010frankencamera}; this was an effort at the time
to create an API that allowed the user to express the different options that
cameras needed.
