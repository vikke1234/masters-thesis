\chapter{History and Background\label{history}}
\section{Background}
This section will introduce the necessary concepts needed for understanding the
thesis.

\subsection{CMOS Sensors}


\section{History}
In the early days cameras were quite simple, there effectively was just the
"press button" followed by receiving a picture. This process did not allow for
much customization. This was the case with most cameras such as webcams where
the camera sensors were "smart sensors". This meant that the . Even the
integrated laptop ones were just USB devices. Linux had initially had lacking
support for this, but in the early 2000s Video For Linux (V4L, later V4L2 for
version 2) was developed. This covered most of the use cases at the time.

% TODO: cite frankencamera
In 2009 the Nokia N900 was released, this was a Linux based phone that was not
like most cameras at the time. It provided interfaces for customizing just
about everything. From ISPs to Image Processing Algorithms (IPA's), this meant
that the current way camera API's worked was no longer sufficient. Enter
Frankencamera\cite{adams2010frankencamera}; this was an effort at the time
to create an API that allowed the user to express the different options that
cameras needed.
