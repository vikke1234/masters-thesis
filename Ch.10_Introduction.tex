\chapter{Introduction\label{intro}}
With the rise of mobile phones, robotics, and small cameras in general,
computational photography has become significantly more important. Because
devices such as mobile phones have such small apertures, the camera does not
receive enough light. This makes images dark and blurry, computational
photography has had a significant effect in the quality of the images as
captured by the image sensor. To improve the picture quality, one approach for
low light photography on mobile devices is to use burst photography and combine
multiple images into a single one\cite{burstPhotMobile}. These use cases will
be covered in a later chapter in detail.

Cameras today are very advanced pieces of hardware, with a long and complex
pipeline from the sensor to the application. The field itself is very
secretive, few manufacturers of either the camera sensor or the processing
chips disclose how their hardware works. Reverse engineering is very
cumbersome, especially in hardware. Few companies will say on a detailed level
what their hardware does because they are often violating each others patents~
\cite{experimentalCompPhot}, if you do not say what you do; you can not be
sued. This increases the secrecy in the field, given these constraints there
are few texts that will give an comprehensive view into the field. This thesis
aims to give an overview of what each part of a camera system does and give a
brief introduction into embedded cameras.

In addition to hardware and software advancement, \textit{Artificial Intelligence} (AI)
has become very common to tackle problems in for example super resolution~
\cite{yangDeepLearningSingle2019, delbracio2021mobile}. Where deep
learning etc. have been used. This thesis will not cover AI uses very deeply
only giving a brief overview of what has been done, instead we will have a
focus on automotive, mobile, and hardware.

We will cover sensors, image signal processors, camera APIs, and finally some
uses. Although Windows does have camera support with their media API
(\textit{Video for Windows}, later DirectX and today \textit{Microsoft Media
Foundation}\footnote{\href{https://learn.microsoft.com/en-us/windows/win32/medfound/microsoft-media-foundation-sdk}{Media
foundation docs}}), it is not a focus in this thesis. We will only be focusing
on cameras on Linux.
