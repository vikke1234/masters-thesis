% \begin{abstract}{finnish}

% Tämä dokumentti on tarkoitettu Helsingin yliopiston tietojenkäsittelytieteen osaston opin\-näyt\-teiden ja harjoitustöiden ulkoasun ohjeeksi ja mallipohjaksi. Ohje soveltuu kanditutkielmiin, ohjelmistotuotantoprojekteihin, seminaareihin ja maisterintutkielmiin. Tämän ohjeen lisäksi on seurattava niitä ohjeita, jotka opastavat valitsemaan kuhunkin osioon tieteellisesti kiinnostavaa, syvällisesti pohdittua sisältöä.


% Työn aihe luokitellaan
% ACM Computing Classification System (CCS) mukaisesti,
% ks.\ \url{https://dl.acm.org/ccs}.
% Käytä muutamaa termipolkua (1--3), jotka alkavat juuritermistä ja joissa polun tarkentuvat luokat erotetaan toisistaan oikealle osoittavalla nuolella.

% \end{abstract}

\begin{otherlanguage}{english}
\begin{abstract}
In todays technological landscape, cameras have transformed into intricate
systems that blend advanced hardware with cutting-edge computational
techniques. Central to these systems are Image Signal Processors (ISPs) are
pivotal in refining raw sensor data through algorithms such as auto white
balancing and exposure control. These processors enhance image quality by
correcting color balance and optimizing brightness. The Raspberry Pi ISP
exemplifies how ISPs manage autocontrol algorithms and provide images as we see
them today.

Camera APIs, such as Video For Linux 2 (V4L2) and
libcamera, provide interfaces for developers to interact with camera hardware.
These APIs vary in complexity and functionality, offering different levels of
control over camera operations. V4L2 is noted for its low-level access,
allowing detailed configuration but requiring significant expertise.
Conversely, libcamera offers a more user-friendly approach with higher-level
abstractions. Computational photography techniques have revolutionized image
capture and processing. Burst photography, for example, combines multiple
images to improve quality in low-light conditions. This technique illustrates
the potential of computational methods to overcome physical limitations of
camera hardware.

As \textit{Artificial Intelligence} (AI) advances, its integration into camera systems
promises further enhancements in image processing capabilities. AI-driven
algorithms can improve features such as super resolution and noise reduction,
extending the limits of current technology. The ongoing evolution in camera
architecture and computational photography not only enhances consumer
electronics but also finds applications in fields like autonomous vehicles and
robotics. These advancements highlight the importance of continued research and
innovation in this dynamic field.

\end{abstract}
\end{otherlanguage}
