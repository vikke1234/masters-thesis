% \begin{abstract}{finnish}

% Tämä dokumentti on tarkoitettu Helsingin yliopiston tietojenkäsittelytieteen osaston opin\-näyt\-teiden ja harjoitustöiden ulkoasun ohjeeksi ja mallipohjaksi. Ohje soveltuu kanditutkielmiin, ohjelmistotuotantoprojekteihin, seminaareihin ja maisterintutkielmiin. Tämän ohjeen lisäksi on seurattava niitä ohjeita, jotka opastavat valitsemaan kuhunkin osioon tieteellisesti kiinnostavaa, syvällisesti pohdittua sisältöä.


% Työn aihe luokitellaan
% ACM Computing Classification System (CCS) mukaisesti,
% ks.\ \url{https://dl.acm.org/ccs}.
% Käytä muutamaa termipolkua (1--3), jotka alkavat juuritermistä ja joissa polun tarkentuvat luokat erotetaan toisistaan oikealle osoittavalla nuolella.

% \end{abstract}

\begin{otherlanguage}{english}
\begin{abstract}
    Today, cameras have become increasingly complex, with few resources to get
    an understanding of what happens underneath the hood. This Thesis aims to
    give an overview of how components in a typical camera stack function, an
    introduction to different camera APIs, and finally give an overview of what
    the differences between them are. We will also cover miscellaneous subjects
    like color formats and more.

    \textbf{TODO: add computational photography if added}

\end{abstract}
\end{otherlanguage}
