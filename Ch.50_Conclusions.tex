\chapter{Conclusion\label{section:conclusions}}
In this Thesis we have explored the complex world of modern camera architecture.
The study has provided an in depth overview of modern camera hardware, providing
an analysis how ISPs, autocontrol algorithms, and various camera APIs work.

One of the key takeaways from this is how complex modern cameras are, involving
a very elaborate pipeline from the sensor to produce a final image.
Despite the secrecy with these pipelines, there are great efforts into open
sourcing many of the components. With Raspberry Pi having an open source camera
stack, and libcamera providing the necessary high level support for people to
use on various platforms.

When comparing the APIs, while they largely do the same thing, and due to their
history of being largely based around the android HAL3, they look very much the
same. They still serve slightly different purposes. Camera2 is the camera API
for all Android systems, while libcamrea is the open source initiative that
will hopefully bring vendors to use more standardized ways for camera
configuration. Argus on the other hand is optimized for speed, it allows for
efficient CUDA usage, allowing one to use GPUs in a very efficient manner.

In conclusion, while significant progress has been made in both hardware and
software components. There is still much room for improvement, especially in
terms of standardization and open source collaboration. As computational
photography continues to evolve, we can expect more breakthroughs that will
push the boundaries of what cameras can achieve in terms of quality and
performance. With all the new developments and camera algorithms, at some point
we may need to address the question, can we still call an image from a camera a
"real" image, and not generated content.
